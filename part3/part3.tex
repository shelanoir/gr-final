\documentclass[../gr-final.tex]{subfiles}
\begin{document}
\part{Conclusion}
\chapter{Conclusion}
This chapter marks the end of the graduation thesis ``
Implementation of an automated reasoning system on fuzzy
propositional logic with Linear Symmetrical Hedge algebra as its
truth domain''. The following sections will summarize the thesis,
discuss its limitations, possible improvements and future
direction.
\section{Summarization of the thesis}
\paragraph{} This thesis has summarized the theory of hedge
algebra and the propositional logic based on it, studied the
$\alpha-Resolution$ method and applied it to design and implement
an automated reasoning system. The thesis has achieved its
objectives:
\begin{itemize}
        \item Studied hedge algebra and Linear Symmetrical Hedge Algebra,
                and investigated the propositional logic with
                LSHA being its truth domain
        \item Researched the resolution method on fuzzy
                propositional logic based on LSHA, its new
                progress $\alpha-Resolution$ and the details of
                implementing them.
        \item The main objective of this thesis, an automated
                reasoning system on fuzzy propositional logic
                based on LSHA using $\alpha-Resolution$, was 
                successfully designed and implemented     
\end{itemize}
\paragraph{} That being said, the thesis has achieved its end.
However there are still many limitations, which will be discussed
in the next section.
\section{Limitations}
The main shortcomings of this thesis including:
\paragraph{The system's limitations:}
\begin{itemize}
        \item {\bfseries Performance:} This system is not a high-performanced
                one, for the lack of time to invest on
                optimization, and mostly because resolution on
                propositional logic has never been a
                high-performance method of automated reasoning.
        \item {\bfseries User Interface:} The system is not exactly
                user-friendly in this regard. Simple and
                essential actions like adding or removing a
                clause are all rather too cumbersome to be useful
        \item {\bfseries Scalability:} Since the performance is not that high
                to begin with, this system is best suit to only
                small to medium knowledge base. This is also not
                very ``scalable'' in the sense of expressiveness,
                because propositional logic can't quantify over other objects 
                so this system can only express so much with its
                logic.                
\end{itemize}
\paragraph{Theoretical research's limitation}
\begin{itemize}
        \item The first limitation of this thesis lies with the flaw 
                of performing resolution with
                $\alpha-Resolution$ is that, the ``confidence''
                value as computed is not really that reliable.
                For instance, suppose we have A1 = $A^{\text{Very
                Very Very True}}$,
                A2 = $A^{\text{More True}}$ and A3 = $A^{True}$, with Very
                and More are both positive, and Very $\ge$ More.
                Now assume we would like to prove A1 and A3 using
                resolution, 
                knowing that A is really More True. Give A2
                maximum confidence and do the usual resolution,
                we will discover that A3 would have the
                confidence of True, while A1 would have the
                confidence of More True. Which is not really
                convincing, since A3 should be much closer to the
                truth (A2) than A1 is.
        \item Resolution on fuzzy logic based on LSHA, and $\alpha-Resolution$ 
                in particular, only computes the confidence value
                of the proof, while in LSHA there are possible
                ways to computes a truth value for a
                proposition.
        \item Another limitation of this thesis is that, this
                thesis didn't consider the much more expressive
                first-order logic, and only studied
                propositiontal logic. This also led to practical
                problems as discussed above.
\end{itemize}
\section{Future direction}
\paragraph{} The limitations discussed above however make several
points for plausible future direction:
\begin{itemize}
        \item Finding new ways to improve resolution on
                propositional logic based on LSHA. 
        \item Considering possible application of
                $\alpha-Resolution$ to other variations of hedge
                algebra.
        \item Investigating possible improvement for
                $\alpha-Resolution$ and resolution in general for
                other kinds logic, based on other kinds of hedge
                algebra               
\end{itemize}
\end{document}
