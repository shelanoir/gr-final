\documentclass[../gr-final.tex]{subfiles}
\begin{document}
\part{Problem definition, Background theory and Implementation Approaches}
\chapter{Introduction}
{\huge //TODO}
\section{Hedge Algebra and Fuzzy logic}
\subsection{Traditional fuzzy logic}
\subsection{Linguistic hedge}
\subsection{Hedge Algebra: an approach to linguistic hedge and
  linguistic logic reasoning}
\section{Problem Definition}
\chapter{Theoretical Background}

\subfile{part1/sec2-1.tex}
\subfile{part1/sec2-2.tex}

\chapter{Implementation Approaches} 
\section{Functional Programming paradigm}
{\huge //TODO}
%\subsection{}
%\subsection{}
\section{Software architecture}
\subsection{Self-contained application} The system is self-contained,
not dependent on the services of any other software component (save
for the operating systems), and does not provide services for any
other software component. 
\subsection{Database-centric architecture} The system is data-centric
in nature, makes use of a database management system (an embedded DBMS
instead of a DBMS server in this case, to make the application
self-contained).
\subsection{Rule-based system} To be more precise, this system is a
platform for both creating knowledgebase and reasoning using the knowledgebase.

\section{Data Modeling Methodology}
\paragraph{Top-down data modeling: } Top-down data modeling methodology enforces that the
design decision of data model is made before the actual
implementation, and the implementation must proceed accordingly. For a
reasoning system like this, its entities and their relationships can
be made quite obvious with just a little upfront thinking, so it is
very appropriate to use top-down methodology.
\end{document}
