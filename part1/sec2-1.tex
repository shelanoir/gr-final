\documentclass[part1.tex]{subfiles}
\begin{document}

\section{Hedge algebra}
\subsection{Definition}
{\bfseries Linguistic hedge:} a unary operation on linguistic value, changing the linguistic value's meaning.
\\
{\bfseries Hedge algebra:} Let \(X\) be a linguistic variable and its domain is \(X = Dom(X)\). A hedge 
algebra \(AX\) in respect to \(X\) is a 4-tuple \(AX = (X,G,H,\le)\) where G is the set of spanning
elements, H is the set of hedges and \(\le\) is the sematic ordering relation on X.
The set G usually has positive, negative and neutral spanning elements. In practice, G usually has only
positive and negative spanning elements e.g \{True,False\}, \{High, Low\}.\\

A hedge algebra AX satisfies the following  axioms:\\
{\indent
	(1) Hedges either increase or decrease the effect of other hedges including itself, and it is 
then positive or negative w.r.t. the other.\\}
{\indent
	(2) If \(u\notin H(v)\) and \(v\notin H(u)\) then \(\forall x\in H(u), x\notin H(v)\) and vice versa. 
	Furthermore if u and v cannot be compared then so are x,y for all \(x\in H(u), y\in H(v)\).
}\\
{\indent
	(3) \(x \not = hx, x \notin H(hx). h\neq k, hx<kx\) then \(h'hx\le k'kx\forall h,h',k,k'\in
	H.\) Furthermore, if \(hx \neq kx\) then hx and kx are independent w.r.t. to each other.
}\\
{\indent
	(4) \(u \in v, u \le v\implies u\le hv\) for \(\forall h\) and vice versa.
}
\\\\
(1) means exactly what it says.\\
(2) means that if two vague concepts are really different, then they form seperated concept categories,
which means they don't have any common meaning.\\
(3) means that each hedge has its own meaning and defines its own concept category.\\
(4) means that a hedge only modifies the vague concept's meaning. It preserves the ordering relation
of the vague concept's meaning w.r.t. other vague concepts.\\
\subsection{Properties}
	{\bfseries Semantic heredity:} When a hedge affects the meaning of a linguistic value, it only
	increases or decreases a little bit the meaning of that value. The resulting value inherits
	most of its parent's meaning i.e the parent's comparability is preserved: 
	if hx \(\le\) kx then H(hx) \(\le\) H(kx). Generally, \(\forall (u,v \in X, 
	\sigma_1 = h_n..h_1, \sigma_2 = k_m..k_1, h_i,k_j \in H)\), we have:
	\(u \le v \implies \sigma_1u \le \sigma_2v\)\\
	{\bfseries Theorem 1:} Let AX = (X,G,H,\(\le\)). The following statements hold:\\
\indent (1) If x \(\in\) X is a fixed point of an operation h in H, i.e. hx = x, then it is a fixed
	 point of the others.\\
\indent (2) If x = \(h_n \cdots h_1 u\), and there exists an index \(i<n\) such that \(h_i \cdots h_1 u\)        is another representation (canonical) of x w.r.t. u and \(h_j x = x, \forall j\ge i\).\\
\indent (3) For any \(h, k\in H\), if \(hx\le kx\) and \(h\neq k\) then \(hx< kx\).\\
\indent \(\implies\) The canonical representation of any linguistic value w.r.t. another value is unique.	\\
\indent {\bfseries Theorem 2:} Let \(x = h_n \cdots h_1 u\) and \(y = k_m \cdots k_1 u\) be two arbitrary
	canonical representations of x and y w.r.t. u, respectively. Then there exists an index
	\(j \le min{n, m} + 1\) such that \(h_i = k_i, \forall i < j\), and\\
\indent (1) \(x < y \iff h_j x_j < k_j x_j , where x_j = h_j-1 \cdots h_1 u\)\\
\indent (2) \(x = y \iff n = m = j and h_j x_j = k_j x_j\)\\
\indent (3) x and y are incomparable \(\iff h_j x_j\) and \(k_j x_j\) are incomparable\\
\indent \(\implies\) This theorem provides us the mean to compare two linguistic value in a hedge 
algebra. 

\subsection{Linear symmetrical algebra}
The spanning set G usually has two comparable linguistic values. For example, we have False \(<\) True
for truth value. A hedge algebra that has only two primitive linguistic values is called a symmetrical
hedge algebra. The spanning set is denoted \(G = \{c^-, c^+\}, c^+, c^-\) are positive and negative 
spanning element, respectively.\\\\

The set H of hedges can also be partitioned into two seperated subsets \(H^+ = \{h|hc^+ > c^+\} = 
\{h|hc^- < c^-\}, H^- = \{h|hc^- > c^-\} = 
\{h|hc^+ < c^+\}\). Any two hedges in \(H^+\) (or \(H^-\)) are comparable to each other, and each hedge in \(H^+\) is converse to a hedge in \(H^-\) and vice versa. A hedge algebra \(AX = (X,\{c^+,c^-\},H^+\cup
H^-, \le)\) is linear symmetrical iff \(H^+\) and \(H^-\) are linearly ordered. It is easy to see
that X is also linearly ordered by \(\le\).\\\\

Let I \(\notin\) H be the {\em identity hedge}: \(\forall x \in X, Ix = x\). I is smaller than any other
hedge in \(H^+\) and \(H^-\).\\\\

If \(H \neq {\o}, H(c^+), H(c^-)\) are infinite then \(inf(c^+) = sup(c^-) = W, inf(c^-) = 0, sup(c^+) = 1\). W is call the neutral element. We have: \(0<c^-<W<c^+<1\). Denote the linguistic value domain as
\(\bar X = X \cup \{0,W,1\}\)\\\\

From that we define:
\begin{itemize}
\item Let \(x = \sigma c, \sigma \in H^*, c \in \{c^+, c^-\}. y \text{is the negation of x, denoted}
	y = -x \text{ if } y = \sigma c', \{c,c'\} = \{c^+, c^-\}. -0 = 1, -1 = 0, -W = W\).
\item \(x,y \in \bar X\) then \(x \wedge y = min(x,y), x \vee y = max(x,y)\)
\end{itemize}


\subsection{Rule of moving hedges}
\indent (RT1): \(\frac{((P,hu),\sigma <True|False>)} {((P,u),\sigma h<True|False>)}\)
\\
\\
	(RT2): \(\frac{((P,u),\sigma h<True|False>)} {((P,hu),\sigma <True|False>)}\)
\section{A propositional logic with truth value domain based on LSA}
\subsection{Syntax}
\indent {\bfseries The alphabet of logic} consists of the following classes of symbol:\\
\indent + Propositional symbols: A,B,C \ldots\\
\indent + Linguistic value symbols: a,b,c \ldots\\
\indent + Constant symbols: 0,1,W\ldots\\
\indent + Logical connectives:\(\vee,\wedge,\to,\neg\) \ldots\\
\indent + Auxiliary symbols: ',', '(', ')'\ldots\\ \\
{\bfseries Literal:} A string \(A^a\), where \(A\) is a propositional symbol and \(a\) is a truth
value in \(A\)'s domain.\\\\
 {\bfseries Formula:}\\
 \indent + A literal is a formula\\
 \indent + F is a formula then \( (\neg F\)) is also a formula\\
 \indent + F and G are formulae then \( (F\vee G), (F\wedge G), (F \to G)\) are also formulae\\
 \indent + Only strings generated by the above rules are formulae\\
 \indent ( {\bfseries Precedence of operations:} \(\neg\)  >>  \(\to\)  >>  \(\wedge\)  >>  \(\vee\))\\

\subsection{Semantic}
\subsubsection{Interpretation}
An {\bfseries Interpretation} I:\{A = a1\} of the literal \(A^{a2}\) is the mapping associate the formula
with a value of the truth value domain.\\

{\bfseries Logical connectives' semantics:} let a and b be some linguistic value. Then:\\
\indent + \(a \vee b = max(a,b)\)\\
\indent + \(a \wedge b = min(a,b)\)\\
\indent + \(\neg a = \text{symmetrical value of } a\)\\
\indent + \(a \to b = max(\neg a,b)\)\\
\(\implies \wedge\) and \(\vee\) are G{\"o}del's T-norm and T-conorm.\\
Let T(A) be A's truth value under an arbitrary interpretation.\\
{\bfseries Truth value of formulae:}\\
\indent + If A is a literal, T(A) is determined by the interpretation\\
\indent + Let A and B be two formulae:\\
\indent \indent (1) \(T(A\vee B) = T(A) \vee T(B)\)\\
\indent \indent (2) \(T(A\wedge B) = T(A) \wedge T(B)\)\\
\indent \indent (3) \(T(A\to B) = T(A) \to T(B)\)\\
\indent \indent (4) \(T(A\neg B) = T(A) \neg T(B)\)
\\\\
Let \(T_{I:\{A = a1\}}(A^{a2})\) the value of that interpretation. Then:\\
\indent + \(T_{I:\{A = a1\}}(A^{a2}) = a1 \wedge a2 \) if a1, a2 \(\ge W\)\\
\indent + \(T_{I:\{A = a1\}}(A^{a2}) = \neg (a1 \vee a2) \) if a1, a2 \(<\) W\\
\indent + \(T_{I:\{A = a1\}}(A^{a2}) = \neg a1 \vee a2 \) if a1 \(\ge\) W, a2 \(<\) W\\
\indent + \(T_{I:\{A = a1\}}(A^{a2}) =  a1 \vee \neg a2 \) if a1 \(<\) W, a2 \(\ge\) W\\

\subsubsection{Satisfiable, unsatisfiable, false}
{\bfseries Definition:} Let S be a formula, I be an arbitrary interpretation:\\
\indent + S is false under I if \(T_I(S) < W\)\\
\indent + S is unsatisfiable if it is contradicted to every interpretation\\ 
\indent + S is satisfiable if \(\exists I: T_I(S) \ge W\). I is then called a model of S, denoted \(I \models S\)\\
\indent + S is a tautology if \(\forall I: T_I(S) \ge W\)\\
\indent \(\implies)\) Collorary: A formula A is a tautology iff \(\neg\)A is unsatisfiable.
\subsubsection{Logical equivalences. Functional complete set of logical connectives}
\indent Two formula A abd V are said to be equivalent (\(A \equiv B\)) if A and B have the same \\
indent truth value for every interpretation I.\\
\indent Some well-known logical equivalences:
\begin{align*}
+ A \to B \equiv \neg A \vee B\\
+ A \to B \equiv \neg A \vee B\\
+ A \to B \equiv \neg A \vee B\\
+ A \to B \equiv \neg A \vee B\\
+ A \to B \equiv \neg A \vee B\\
+ A \to B \equiv \neg A \vee B\\
+ A \to B \equiv \neg A \vee B\\
+ A \to B \equiv \neg A \vee B\\
+ A \to B \equiv \neg A \vee B\\
\end{align*}
\indent {\bfseries Functionally complete set of logical connectives} is one which can be used to express all possible logical functions \(f:TV^n \to TV\)\\

\subsubsection{Conjunctive normal form}
A formula expressed as a conjunction of formulae, where these formulae are in turn expressed as
a disjunction of literal, is said to be in {\em Conjunctive normal form}.\\
{\bfseries Theorem:} Every formula in our propositional logic is logically equivalent to a CNF
formula.\\
Algorithm to transform a formula into its CNF:\\\\
\indent- Eliminate \(\to\) and \(\leftrightarrow\) :
\begin{align*}
	&F \to G \underset{CNF}{\implies} \neg F \vee G\\
	&F \leftrightarrow G \underset{CNF}{\implies} (\neg F \vee G) \wedge (\neg G \vee F)
\end{align*}
\indent- Move \(\neg\) inward: 
\begin{align*}
	&\neg (F \vee G) \underset{CNF}{\implies} \neg F \wedge \neg G
\end{align*}
\indent- Elimitate double negation:
\begin{align*}
	&\neg \neg F \underset{CNF}{\implies} F
\end{align*}
\indent- Applying distributivity law:
\begin{align*}
	&F \wedge (G \vee H) \underset{CNF}{\implies} (F \wedge G) \vee (F \wedge H)\\
	&F \vee (G \wedge H) \underset{CNF}{\implies} (F \vee G) \wedge (F \vee H)
\end{align*}	
\indent- Rewrite redundant formula:
\begin{align*}	
	&F \vee F \underset{CNF}{\implies} F\\
	&F \wedge F \underset{CNF}{\implies} F
\end{align*}
\end{document}
