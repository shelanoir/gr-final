\documentclass[12pt,fleqn,a4paper]{report}
\usepackage{my}
\usepackage[vietnam, english]{babel}
\usepackage[T5]{fontenc}
\date{}
\usepackage{pdfpages}
\begin{document}
%\large
%\title{{\huge Graduation Research Thesis}\\
%Linguistic Logic Reasoning:\\
%Implementation of Alpha Resolution on fuzzy propositional logic based on Linear Symmetrical Hedge algebra\\
%Implementation of an automated reasoning system on fuzzy propositional
%logic with Linear
%Symmetrical Hedge algebra as its truth domain.
%}
%\author{Instructor: Tran Dinh Khang\\Student: Ngo Nhat Anh, ID: 20090102}
%\maketitle
\includepdf{title-test.pdf}
\includepdf{objectives.pdf}
%\begin{figure}[H]
%        \includegraphics[scale=0.9]{title-test}
%\end{figure}
%test
%\begin{align}
%e^i\pi\\ A \in B \\B \ni A \\A \not\ni B \\M > N \\\bar{A} \equiv B \cap \bar{C}
%\end{align}
%endtest
%
%

\begin{abstract}
{\bfseries To address a problem with expressing the ordering
	of vague concepts in traditional fuzzy logic, a kind
	of algebraic structure called hedge algebra was
	proposed as an alternative domain for fuzzy truth
	value and linguistic vague concepts. The progress in
	automated reasoning on fuzzy logic based on hedge
	algebra has been growing substantial, and promises many
        potential practical application for hedge algebra.\\\\
	With that as the motivation, this thesis studies and
        summarizes some key points in fuzzy propositional logic
        with Linear Symmetrical Hedge Algebra as its truth domain, 
        investigates the resolution reasoning method for
        this logic, along with $\alpha-resolution$, a recent
        development of this resolution method.\\\\
        With that as the theoretical foundation, this thesis proposes the design and
        implementation of an automated reasoning system applying 
	the $\alpha-resolution$ inference method, which might be used for 
        small to medium knowledge-based application.

}
\end{abstract}

\renewcommand{\abstractname}{Tóm tắt nội dung đồ án}
\begin{abstract}
        {\bfseries Để giải quyết vấn đề trong logic mờ truyền thống
                về biểu diễn thứ tự của các khái niệm mờ, một cấu
                trúc toán học có tên gọi Đại số gia tử được đề
                xuất như một lựa chọn khác bên cạnh tập mờ 
                cho miền giá trị chân lý và các khái niệm ngôn
                ngữ mờ. Nghiên cứu về suy diễn tự động cho logic
                mờ trên miền giá trị chân lý là đại số gia tử đã
                có tiến triển đáng kể, và mở ra nhiều tiềm năng
                cho các ứng dụng thực tế sử dụng ĐSGT.\\\\
                Với những thực tế nêu trêu làm động lực, đồ án
                tốt nghiệp này nghiên cứu và tổng kết một số điểm
                chính của logic mờ với miền giá trị chân lý là     
                Đại số gia tử Đối xứng Tuyến tính, tìm hiểu về
                phương pháp suy diễn tự động bằng hợp giải cho
                loại logic nói trên, cùng với Hợp giải $\alpha$,
                một kết quả mới về phương pháp hợp giải.\\\\
                Dựa vào nền tảng lý thuyết đó, đồ án đề
                xuất thiết kế và cài đặt cho một hệ thống suy
                diễn tự động áp dụng phương pháp Hợp giải
                $\alpha$, hệ thống này có thể sử dụng cho các ứng
                dụng tri thức vừa và nhỏ.}
                
%blah blah blah huehuehuehueheuhue        
%hê hê hê
\end{abstract}
\listoftables
\listoffigures
\tableofcontents
\thispagestyle{fancy}
\subfile{part1/part1.tex}
\subfile{part2/part2.tex}
\subfile{part3/part3.tex}
\end {document}
